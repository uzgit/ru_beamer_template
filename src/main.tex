\documentclass[aspectratio=169]{beamer}
 
\usepackage[utf8]{inputenc}
\usepackage{lipsum}

%%%%%%%%%%%%%%%%%%%%%%%%%%%%%%%%%%%%%%%%%%%%%%%%
% Set Helvetica Font in Text and Math in LaTeX %
%%%%%%%%%%%%%%%%%%%%%%%%%%%%%%%%%%%%%%%%%%%%%%%%
\usepackage[helvet]{sfmath}
\usepackage[scaled=1]{helvet}
\usepackage[T1]{fontenc}
\renewcommand\familydefault{\sfdefault} 
\everymath={\sf}
\usepackage{fontspec}

\usepackage{subcaption}

\hypersetup{
    colorlinks=true,
    linkcolor=black,
    filecolor=magenta,      
    urlcolor=cyan,
    pdftitle={Overleaf Example},
%    pdfpagemode=FullScreen,  % uncomment this line to make the pdf open in fullscreen mode by default
}

% use settings files
\usetheme[progressbar=frametitle]{metropolis}

%%%%%%%%%%%%%%%%%%%%%%%%%%%%%%%%%%%%%%%%%%%%%%%%%%%%%%%%%%%%%%%%
% Color settings
\definecolor{ru_white}{HTML}{FFFFFF}
\definecolor{ru_red}{HTML}{ed1c24}
\definecolor{black}{HTML}{000000}
\definecolor{gray}{HTML}{A9A9A9}
\definecolor{light_gray}{HTML}{E0E0E0}
\setbeamercolor{frametitle}
{
  use=palette primary,
  parent=palette primary,
  fg = ru_white,
  bg = ru_red
}
% fg means "foreground" and bg means "background"
\setbeamercolor{title separator}{ fg=ru_red }
\setbeamercolor{section bar}{ fg=ru_red, bg=ru_red }
\setbeamercolor{progress bar in section page}{
%   use=progress bar,
%   parent=progress bar
    fg=black,
    bg=ru_red
}

\setbeamercolor{normal text}{%
  fg=black,
  bg=white
}
\setbeamercolor{alerted text}{%
  fg=ru_red,
}
\setbeamercolor{example text}{%
  fg=black,
}

% \addtobeamertemplate{section page}{}{\vspace{0.35\paperwidth}\hspace*{0.78\paperwidth}\includegraphics[width=.1\textwidth]{images/ru_logo_transparent.png}}
% \addtobeamertemplate{section page}{
% \begin{center}\includegraphics[width=.1\textwidth]{images/ru_logo_transparent.png}\end{center}
% }{}
% \addtobeamertemplate{section page}{}{\titlegraphic}

\setbeamercolor{progress bar in head/foot}{ fg=black, bg=gray } % use this line to enable the progress bar
% \setbeamercolor{progress bar in head/foot}{ fg=black, bg=black } % use this line to have a divider instead of a progress bar

% \setbeamertemplate{frame numbering}{} % uncomment this line to remove slide numbers
\titlegraphic{\hfill\includegraphics[height=2.0cm]{images/ru_logo_transparent.png}}
% \sectiongraphic{\hfill\includegraphics[height=2.0cm]{images/ru_logo_transparent.png}}
\logo{\includegraphics[width=.1\textwidth]{images/ru_logo_transparent.png}\vspace{-0.05\paperwidth}\hspace*{.04\paperwidth}} % uncomment this line to include the RU logo on each slide

\setbeamertemplate{itemize items}[default]

\setbeamerfont{bibliography entry author}{size=\tiny,series=\normalfont}
\setbeamerfont{bibliography entry title}{size=\tiny,series=\bfseries}
\setbeamerfont{bibliography entry location}{size=\tiny,series=\normalfont}
\setbeamerfont{bibliography entry note}{size=\tiny,series=\normalfont}

\makeatletter
\setlength{\metropolis@progressinheadfoot@linewidth}{1pt}               % thickness of progress bar on normal pages
\setlength{\metropolis@titleseparator@linewidth}{0.5pt}                 % thickness of separator on title page
\setlength{\metropolis@progressonsectionpage@linewidth}{0.5pt}          % thickness of progress bar on section page


\setbeamertemplate{frametitle}{%
    \vspace*{-0.12cm}
     \begin{beamercolorbox}[wd=\paperwidth,ht=2ex,dp=0.8ex,left]{frametitle}%
    %   \hspace*{2ex}\insertframetitle%
        % \vspace*{\fill}
        \hspace*{2ex}\insertframetitle%
        % \vspace*{\fill}
     \end{beamercolorbox}}
\input{configuration_files/quotations.tex}

% Other packages
\usepackage{enumitem}
\setitemize{label=\usebeamerfont*{itemize item}%
  \usebeamercolor[fg]{itemize item}
  \usebeamertemplate{itemize item}}

\usepackage{amssymb}
\newcommand*{\QEDA}{\hfill\ensuremath{\blacksquare}}

\usepackage{datetime}
\newdateformat{specialdate}{\twodigit{\THEDAY} \ \monthname[\THEMONTH] \THEYEAR}
%%%%%%%%%%%%%%%%%%%%%%%%%%%%%%%%%%%%%%%%%%%%%%%%%%%%%%%%%%%%%%%%%

\newcommand{\nologo}{\setbeamertemplate{logo}{}}
\newcommand{\yeslogo}{\logo{\includegraphics[width=1.5cm]{./images/ru_logo_transparent}}}

\yeslogo

\newif\ifpause
\pausetrue
%\pausefalse
\newcommand{\mypause}{\ifpause \pause \fi}

%Information to be included in the title page:
\title{RU Beamer Template}
\author{Author}
\institute{Reykjavík University\\Department of Computer Science}

\date{\specialdate\today}

\begin{document}

\maketitle

\section{Introduction}

\begin{frame}{Introduction}
    This is a simple, minimalistic presentation template made using \LaTeX's Beamer package with the Metropolis theme, and with the colors, fonts, and logo for Reykjavík University.
    
    You can adjust the position of the logo on normal frames with the \texttt{logo} command in the \texttt{configuration\_files/beamer\_theme.tex} file.
    
\end{frame}

% each frame is a slide
\begin{frame}{General Text: Normal, Italicized, Bold, Bold-Italicized}
    \begin{itemize}
        \item Lorem ipsum dolor sit amet consectetuer adipiscing elit.
        \item \textit{Lorem ipsum dolor sit amet consectetuer adipiscing elit.}
        \item \textbf{Lorem ipsum dolor sit amet consectetuer adipiscing elit.}
        \item \textbf{\textit{Lorem ipsum dolor sit amet consectetuer adipiscing elit.}}
    \end{itemize}
    
\end{frame}

\begin{frame}{Example Math}
    Testing math fonts.
    
    $$\int_{1}^{\infty} \frac{1}{x^2} dx = \lim_{b \rightarrow \infty} \int_{1}^{b} \frac{1}{x^2} dx = \lim_{b \rightarrow \infty} \left( - \frac{1}{b} + \frac{1}{1} \right) = 1$$
    
    \[
    \left[
	\begin{matrix}
		-2 & 1 & 0 & 0 & \cdots & 0  \\
		1 & -2 & 1 & 0 & \cdots & 0  \\
		0 & 1 & -2 & 1 & \cdots & 0  \\
		0 & 0 & 1 & -2 & \ddots & \vdots \\
		\vdots & \vdots & \vdots & \ddots & \ddots & 1  \\
		0 & 0 & 0 & \cdots & 1 & -2
	\end{matrix}
	\right]
    \]
    
\end{frame}

\begin{frame}{More Math}
    \begin{eqnarray}
    	\omega_1 & = &
    	\frac{\partial w}{\partial y}-\frac{\partial v}{\partial z}\,,
    	\nonumber  \\
    	\omega_2 & = & 
    	\frac{\partial u}{\partial z}-\frac{\partial w}{\partial x}\,,
    	\label{eqcurl}  \\
    	\omega_3 & = & 
    	\frac{\partial v}{\partial x}-\frac{\partial u}{\partial y}\,.
    	\nonumber
    \end{eqnarray}
    
    \begin{eqnarray*}
    	(p\wedge q)\vee(p\wedge\neg q) & = & p\wedge(q\vee\neg q)
    	\quad\text{by distributive law}  \\
    	 & = & p\wedge T \quad\text{by excluded middle}  \\
    	 & = & p \quad\text{by identity} \;\blacksquare
    \end{eqnarray*}
    
\end{frame}

\begin{frame}{}
    You can make frames with no title bar by simply leaving the title argument blank.

\end{frame}

\begin{frame}{Column Environments}
    \begin{columns}
        \begin{column}{0.47\textwidth}
            \includegraphics[width=\textwidth]{example-image-a}
        \end{column}
        \begin{column}{0.5\textwidth}
            \footnotesize
            Use the \texttt{column} environment to divide slides horizontally.
        \end{column}
    \end{columns}
\end{frame}

\begin{frame}{Activate/deactivate page logo}
	Use \texttt{\textbackslash nologo} or \texttt{\textbackslash yeslogo}
	\emph{before} creating a new frame
	to enable or disable the
	logo in the bottom right of the frame.
\end{frame}

\nologo
\begin{frame}{Image and Text Side by Side}
    \begin{columns}
        \begin{column}{0.47\textwidth}
            \includegraphics[width=\textwidth]{example-image-b}
        \end{column}
        \begin{column}{0.5\textwidth}
            \footnotesize
		\lipsum[10]
        \end{column}
    \end{columns}
\end{frame}

\begin{frame}{Image and Text Side by Side (Other Way)}
    \begin{columns}
        \begin{column}{0.5\textwidth}
            \footnotesize
		\lipsum[11]
        \end{column}
        \begin{column}{0.47\textwidth}
            \includegraphics[width=\textwidth]{example-image-c}
        \end{column}
    \end{columns}
\end{frame}

\begin{frame}{Quotations}
    Configure the settings for quotations in the \texttt{configuration\_files/quotations.tex} file.
\end{frame}

\begin{frame}{Quotation}
    \begin{shadequote}[r]{Speaker}
	    \lipsum[4]
    \end{shadequote}
\end{frame}

\yeslogo
\begin{frame}{Citations}

    \begin{itemize}
        \item This is a cited sentence.\cite{einstein}
        \item This is a cited sentence.\cite{dirac}
        \item This is a cited sentence.\cite{knuthwebsite}
        \item This is a cited sentence.\cite{knuth-fa}
    \end{itemize}
    
\end{frame}

\setbeamertemplate{bibliography item}[triangle]
\begin{frame}{References}
        \bibliographystyle{plain}
        \bibliography{references}
\end{frame}

\end{document}
